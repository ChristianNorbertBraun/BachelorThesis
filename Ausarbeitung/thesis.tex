\documentclass[12pt,oneside,a4paper,parskip]{scrbook}
\usepackage[utf8]{inputenc}
\usepackage{csquotes}
\usepackage[ngerman]{babel}
\usepackage{floatflt} 
\usepackage{subfigure}
\usepackage[pdftex]{graphicx}
\usepackage[hidelinks]{hyperref}
\usepackage{color}
\usepackage{amssymb}
\usepackage{textcomp}
\usepackage{nicefrac}
\usepackage{pdfpages}
\usepackage{float} 
\usepackage{pdflscape}
\usepackage{subfigure}
\usepackage{pdfpages}  
\usepackage[verbose]{placeins} 
\usepackage[nouppercase,headsepline,plainfootsepline]{scrpage2}
\usepackage{listings}		
\usepackage{xcolor}			
\usepackage{color}			
\usepackage{caption}		
\usepackage{subfigure}			
\usepackage{epstopdf}		
\usepackage{longtable}  
\usepackage{setspace}
\usepackage{booktabs}
\usepackage[style=numeric]{biblatex}
%\bibliography{literatur2}
\addbibresource{literatur.bib}


%%%%%%%%%%%%%%%%%%%
%% definitions
%%%%%%%%%%%%%%%%%%%
\def\BaAuthor{Christian Norbert Braun}
\def\BaTitle{Einsatz eines Distributed File Systems zur Skalierung eines Banking-Buchungssystems}
\def\BaSupervisorOne{Prof.\ Dr.\ Steffen Heinzl}
\def\BaSupervisorTwo{Prof.\ Dr.\ Peter Braun}
\def\BaDeadline{31.03.2017}

\hypersetup{
pdfauthor={\BaAuthor},
pdftitle={\BaTitle},
pdfsubject={Subject},
pdfkeywords={Keywords}
}

%%%%%%%%%%%%%%%%%%%
%% configs to include
%%%%%%%%%%%%%%%%%%%
\colorlet{punct}{red!60!black}
\definecolor{background}{HTML}{EEEEEE}
\definecolor{delim}{RGB}{20,105,176}
\colorlet{numb}{magenta!60!black}

\definecolor{gray}{rgb}{0.4,0.4,0.4}
\definecolor{darkblue}{rgb}{0.0,0.0,0.6}
\definecolor{cyan}{rgb}{0.0,0.6,0.6}

\definecolor{pblue}{rgb}{0.13,0.13,1}
\definecolor{pgreen}{rgb}{0,0.5,0}
\definecolor{pred}{rgb}{0.9,0,0}
\definecolor{pgrey}{rgb}{0.46,0.45,0.48}

\lstset{
  basicstyle=\ttfamily,
  columns=fullflexible,
  showstringspaces=false,
  commentstyle=\color{gray}\upshape
  linewidth=\textwidth
}

\lstdefinelanguage{json}{
    basicstyle=\normalfont\ttfamily,
    numbers=left,
    numberstyle=\scriptsize,
    stepnumber=1,
    numbersep=8pt,
    showstringspaces=false,
    breaklines=true,
    backgroundcolor=\color{background},
    literate=
     *{0}{{{\color{numb}0}}}{1}
      {1}{{{\color{numb}1}}}{1}
      {2}{{{\color{numb}2}}}{1}
      {3}{{{\color{numb}3}}}{1}
      {4}{{{\color{numb}4}}}{1}
      {5}{{{\color{numb}5}}}{1}
      {6}{{{\color{numb}6}}}{1}
      {7}{{{\color{numb}7}}}{1}
      {8}{{{\color{numb}8}}}{1}
      {9}{{{\color{numb}9}}}{1}
      {:}{{{\color{punct}{:}}}}{1}
      {,}{{{\color{punct}{,}}}}{1}
      {\{}{{{\color{delim}{\{}}}}{1}
      {\}}{{{\color{delim}{\}}}}}{1}
      {[}{{{\color{delim}{[}}}}{1}
      {]}{{{\color{delim}{]}}}}{1},
}

\lstset{language=xml,
  morestring=[b]",
  morestring=[s]{>}{<},
  morecomment=[s]{<?}{?>},
  stringstyle=\color{black},
  numbers=left,
  numberstyle=\scriptsize,
  stepnumber=1,
  numbersep=8pt,
  identifierstyle=\color{darkblue},
  keywordstyle=\color{cyan},
  backgroundcolor=\color{background},
  morekeywords={xmlns,version,type}% list your attributes here
}

\lstset{language=Java,
  showspaces=false,
  showtabs=false,
  tabsize=4,
  breaklines=true,
  keepspaces=true,      
  numbers=left,
  numberstyle=\scriptsize,
  stepnumber=1,
  numbersep=8pt,
  showstringspaces=false,
  breakatwhitespace=true,
  commentstyle=\color{pgreen},
  keywordstyle=\color{pblue},
  stringstyle=\color{pred},
  basicstyle=\ttfamily,
  backgroundcolor=\color{background},
%  moredelim=[il][\textcolor{pgrey}]{$$},
%  moredelim=[is][\textcolor{pgrey}]{\%\%}{\%\%}
}




\begin{document}


%%%%%%%%%%%%%%%%%%%
%% Titelseite
%%%%%%%%%%%%%%%%%%%


\frontmatter
\titlehead{%  {\centering Seitenkopf}
  {Hochschule für angewandte Wissenschaften Würzburg-Schweinfurt\\
   Fakultät Informatik und Wirtschaftsinformatik}}
\subject{Bachelorarbeit}
\title{\BaTitle\\[15mm]}
\subtitle{\normalsize{vorgelegt an der Hochschule f\"{u}r angewandte Wissenschaften W\"{u}rzburg-Schweinfurt in der Fakult\"{a}t Informatik und Wirtschaftsinformatik zum Abschluss eines Studiums im Studiengang Informatik}}
\author{\BaAuthor}
\date{\normalsize{Eingereicht am: \BaDeadline}}
\publishers{
  \normalsize{Erstpr\"{u}fer: \BaSupervisorOne}\\
  \normalsize{Zweitpr\"{u}fer: \BaSupervisorTwo}\\
}

%\uppertitleback{ }
%\lowertitleback{ }

\maketitle


%%%%%%%%%%%%%%%%%%%
%% abstract
%%%%%%%%%%%%%%%%%%%

\section*{Zusammenfassung}
\addcontentsline{toc}{chapter}{Zusammenfassung}

TODO

\section*{Abstract}
\addcontentsline{toc}{chapter}{Abstract}

TODO

\newpage
\chapter*{Danksagung}
\addcontentsline{toc}{chapter}{Danksagung}
Danke an FH und adorsys. An Francis und an Prof.Heinzl. Eventuell auch an Korrekturleser.

%%%%%%%%%%%%%%%%%%%
%% Inhaltsverzeichnis
%%%%%%%%%%%%%%%%%%%
\tableofcontents										



%%%%%%%%%%%%%%%%%%%
%% Main part of the thesis
%%%%%%%%%%%%%%%%%%%
\mainmatter


\chapter{Einführung}\label{ch:intro}
\section{Motivation}
Quellen benötigt!!!!
Banken sind Big Data. Jeden Tag fließen tausende Zahlungsprozesse alleine durch die Systeme deutscher Banken. Der stetige Datenzuwachs birgt zahlreiche Möglichkeiten jedoch auch Herausforderungen. Zur Auswertung werden hoch skalierbare und performante Systeme benötigt. Die aktuell weitläufig in Kernbankensystemen eingesetzten Technologien stoßen hier an ihre Grenzen oder erfordern teure Hardware um den Anforderungen gerecht zu werden. Durch Nutzung der Sharing Economy skalieren Systeme entsprechend der Anforderungen und gewährleisten hohe Stabilität. Diese Vorteile können zu einem wirtschaftlichen Vorteil bei der Realisierung eines Banking Buchungssystems genutzt werden.
\section{Umfeld}
In Zusammenarbeit mit der adorsys GmbH \& Co. Kg. gilt es die Möglichkeiten durch den Einsatz der Sharing Economy für ein Banking Buchungssystem im Rahmen dier Arbeit zu analysieren. adorsys ist eine It Consulting Firma, die auch zahlreiche Firmen im Versicherungs und Banking-Bereich betreut. Dabei wird besonderer Wert auf die Betreuung als Komplettpacket gelegt. adorsys liefert über Beratung, Design und Implementierung auf die Kunden zugeschnittene Lösungen. 
\section{Zielsetzung}
Diese Arbeit diskutiert die Anforderungen an aktuelle Buchungssysteme und stellt die dazu verwendeten Technoligien in Frage. Weg von einem System, welches nach den Möglichkeiten der Technologie gebaut wurde, hin zu einer Lösung, die auf Grundlage der Aufgaben eines Buchungssystems die richtigen Mittel und Methoden sucht. Dabei sollen besonders die Punkte Skalierbarkeit, Ausfallsicherheit und Performance beleuchtet und analysiert werden. 

\chapter{Methodik}
Wenig Emirpie mehr Analytik
\section{Analyse der Ist-Situation}
Analyse von Kernbankensystemen und deren Persistenzschicht. Was kostet das eine Bank im Jahr? 
\section{Analyse und Auswahl eines Distributed File Systems}
Analyse mehrerer DFS in Betrachtung der Anforderungen eines Buchungssystems und Auswahl von SeaweedFS
\section{Entwicklung des Konzepts}
Auf Basis der fehlenden Transaktionen das Konzept entwickeln. Was muss alles abgebildet werden etc.?
\section{Beispielhafte Implementierung}
Simple Beispielimplementierung zum Testen der Performance und der Umsetzbarkeit
\section{Bewertung der Lösung}

\chapter{Wesen und Probleme eines Buchungssystems}
\section{Begriffserklärung}
Was genau ist ein Buchungssystem?
Buchungssystem als Teil eines Kernbankensystems
Double Entry Book Keeping
\section{Bestandteile}
Welche Konten befinden sich darin? 
Aktiv- und Passivgeschäft erläutern
\section{Anforderungen}
Transaktionen erwähnen. Besonders auch auf aktuelle Technologie eingehen (RDBMS). Ausfallsicherheit, Erreichbarkeit, etc.
\section{Probleme}
Sehr teuer (Quelle????)
Schwer zu skalieren
Es wird Transaktionssicherheit auch für Kontoarten gewährleistet, welche gar keine benötigen

\chapter{Dirstributed File System als Backbone für Buchungssysteme}

\section{Funktionsweise}
Verteiltes Datei System. Vielleicht auf Funktionsweise von Hadoop, etc. eingehen. Aber nur Überblick (Paper beinhalten Zusatzinformation).  
In der Regel keine Update Operation => Keine Transaktionen
\section{Anwendungsbereiche}
Alles was mit Big Data zu tun hat. Google zum speichern der Weblinks, Facebook für Bilder, Yahoo, etc.
\section{Vorteile und Nachteile}
Schnell, billig, skalierbar, ausfallsicher, etc.
Keine Transaktionen, eher low level, Analysen sind aufwendig (Map Reduce)

\chapter{Konzept}
\section{Transaktionen aufgeben}
Kann man Transaktionen aufgeben? Wieso lohnt sich das? Was ist mein Ziel damit?
\section{Ausfallsicherheit}
\section{Verwendung des Distributed File Systems}
Wie wird das DFS eingesetzt? Noch keine Konkrete Implementierung angeben.
\section{Aufbau der Anwendung}
RESTfull Service, Kommunikation mit DFS
\section{Erstellen einer Buchung}
Ablauf
\section{Lesen einer Buchung}
Ablauf
\section{Maßnahmen zur Skalierung}
Wegen DFS leicht skalierbar, aber es muss auch auf Locks beim Schreiben von neuen Dateien geachtet werden.


\chapter{Implementierung}
\section{SeaweedFS}
Einsatz von SeaweedFS und wieso?
Wie wird er hier konkret Konfiguriert
\section{golang}
Wieso habe ich golang verwendet? Bezug zu SeaweedFS 
\section{Schnittstelle zu SeaweedFS}
Wieso wurde eine neue Schnittstelle geschrieben?
Worauf war zu achten? Nutzung des Filers (Distributed Filer)
\section{Bibliothek zur Abbildung von Buchungen}
Einführen der fehlenden Abstraktionsschicht für die spätere Anwendung
\section{RESTful Webservice}
Implementierung einer API zum leichten Anlegen und Lesen von Buchungen. Besonderes Augenmerk auf Modularisierung.

\chapter{Evaluierung}
Mal sehen was hier später steht.

\chapter{Ausblick}
Ich bin sehr gespannt.

\begin{lstlisting}[label=lst:java,
				   language=java,
				   firstnumber=1,
				   caption=Beispiel für einen Quelltext]				   

public void foo() {				   
	// Kommentar
}
\end{lstlisting}

\chapter{Zusammenfassung}


\backmatter
%%%%%%%%%%%%%%%%%%%
%% create figure list
%%%%%%%%%%%%%%%%%%%

\listoffigures
\addcontentsline{toc}{chapter}{Verzeichnisse}			

%%%%%%%%%%%%%%%%%%%
%% create tables list
%%%%%%%%%%%%%%%%%%%
\listoftables

%%%%%%%%%%%%%%%%%%%
%% create listings list
%%%%%%%%%%%%%%%%%%%
%\lstlistoflistings
%\addcontentsline{toc}{chapter}{Listings}				

\printbibliography
\addcontentsline{toc}{chapter}{Literatur}				

%%%%%%%%%%%%%%%%%%%
%% declaration on oath
%%%%%%%%%%%%%%%%%%%

\addchap{Eidesstattliche Erklärung}

Hiermit versichere ich, dass ich die vorgelegte Bachelorarbeit selbstständig verfasst und noch nicht anderweitig zu Prüfungszwecken vorgelegt habe. Alle benutzten Quellen und Hilfsmittel sind angegeben, wörtliche und sinngemäße Zitate wurden als solche gekennzeichnet.

\vspace{20pt}
\begin{flushright}
$\overline{~~~~~~~~~~~~~~~~~\mbox{\BaAuthor, am \today}~~~~~~~~~~~~~~~~~}$
\end{flushright}
\end{document}


